\documentclass{article}
\usepackage[utf8]{inputenc}
\usepackage[russian]{babel}
\usepackage{listings}
\usepackage[a4paper, total={6in, 10in}]{geometry}

\title{Документация к генератору кс-грамматики}
\author{Константинова Полина, Павлов Игорь}

\begin{document}

\maketitle

\newpage
\tableofcontents

\newpage

\section{Описание грамматики}
\subsection{Грамматика}
\qquad Программа генерирует КС-грамматики, описывающиеся грамматикой, приведённой ниже:
\begin{lstlisting}
<grammar> ::= <rule> <rule-sep> <grammar-tail>
<grammar-tail> ::= <rule> <rule-sep> <grammar-tail>
<grammar-tail> ::=

<rule> ::= <nonterm> <arrow> <rule-right>

<rule-right> ::= <production> <rule-tail>
<rule-tail> ::= <production-sep> <production> <rule-tail>
<rule-tail> ::=

<production> ::= <epsilon>
<production> ::= <term> <production-tail>
<production> ::= <nonterm> <production-tail>

<production-tail> ::= <term> <production-tail>
<production-tail> ::= <nonterm> <production-tail>
<production-tail> ::=

<term> ::= <term-regex>
<nonterm> ::= <nterm-start> <nterm-regex> <nterm-end>
\end{lstlisting}

Токены, задаваемые пользователем:

\begin{lstlisting}
<rule-sep>
<arrow>
<production-sep>
<nterm-start>
<nterm-end>
<term-regex>
<nterm-regex>
\end{lstlisting}

\subsection{Свойства грамматики}
\justifying
\qquad При разумном выборе пользователем токенов грамматика детерминированная и \verb|LL(1)|. Грамматика \verb|LR(1)|, так как префикс-свойство не выполняется.

\section{Генератор}
\subsection{Возможности}
\qquad Данная программа позволяет генерировать КС-грамматики с пользовательским синтаксисом. Есть возможность генерировать только достижимые нетерминалы, только генерирующие нетерминалы и нетерминалы, не порождающие пустые слова.

\quad Также есть возможность проверки грамматики на \verb|LL(1)|. Если пользовательский синтаксис будет нарушать \verb|LL(1)|, выведется ошибка, которая укажет, какие токены надо заменить.

\quad Грамматика генерируется не в виде строки, а в виде  структуры. Поэтому есть возможность применять алгоритмы анализа КС-грамматик.

\subsection{Установка и запуск}
\qquad Для запуска генератора вам понадобится язык программирования python3 и установщик пакетов pip. Далее склонируйте репозиторий и установите зависимости:

\begin{lstlisting}
git clone https://github.com/pollykon/FLT_lab5
cd FLT_lab5
pip install -r requirements.txt
\end{lstlisting}

\qquad Для работы генератора необходимо создать конфигурационный файл и указать в нем параметры генерации. Для запуска генератора используйте команду:

\begin{lstlisting}
python generator.py < path/to/config.yaml
\end{lstlisting}

\qquad Программа выведет сгенерированную грамматику, а также два флага: являются ли все правила достижимыми и все нетерминалы порождающими.

\qquad Чтобы включить проверку получаемой грамматики на LL(1) свойство, используйте ключ \verb|--check-ll1|:

\begin{lstlisting}
python generator.py --check-ll1 < path/to/config.yaml
\end{lstlisting}

\subsection{Структура проекта}

\begin{lstlisting}
FLT_lab5/
- .gitignore
- additional/
│   - attribute_grammar.pdf
│   - lab_5_dop.pdf
- cfg.py
- configs/
│   - default.yaml
│   - test.yaml
- generator.py
- README.md
- requirements.txt
- utils.py
- validator.py
\end{lstlisting}

\qquad В репозитории в директории \verb|configs| лежат конфигурационные файлы в формате \verb|yaml|. В \verb|validator.py| находится код валидации параметров конфига и проверки получаемой грамматики на LL(1) свойство. В модуле \verb|cfg.py| объявлен класс, описывающий КС-грамматику. Основной модуль - \verb|generator.py|, в нем находится код генератора. Генератор генерирует грамматику в виде экземпляра класса \verb|CFG|.

\subsection{Настраиваемые параметры}
\qquad В репозитории в директории \verb|configs| лежат файлы с пользовательским синтаксисом. Параметры указываются в файле формата \verb|yaml|. Если какие-то параметры пользователь не указал, они загружаются из файла \verb|default.yaml|.

\verb|grammar|:

\verb|nonterminals_range| - (list) диапазон количества нетерминалов. Значение по умолчанию: \verb|[1, 6]|;

\verb|rules_for_nonterminal_range| - (list) диапазон количества правил для нетерминала. Значение по умолчанию: \verb|[1, 3]|;

\verb|only_non_empty_nonterminals| - (boolean) флаг для генерации нетерминалов, не порождающих пустое слово (при значении \verb|True|). Значение по умолчанию: \verb|False|;

\verb|unreachable_nonterminals| - (boolean) флаг для генерации недостижимых нетерминалов (при значении \verb|True|). Значение по умолчанию: \verb|False|;

\verb|only_generating_nonterminals| - (boolean) флаг для генерации только порождающих нетерминалов (при значении \verb|True|). Значение по умолчанию: \verb|False|;

\verb|start_nonterminal| - (string) стартовый нетерминал. Значение по умолчанию: \verb|S|;

\verb|production_separator| - (string) разделитель для продукций. Значение по умолчанию: |;

\verb|arrow| - (string) разделитель для нетерминала и его правила. Значение по умолчанию: \verb|->|;

\verb|rule_separator| - (string) разделитель для правил. Значение по умолчанию \verb|;|;

\verb|epsilon|:

\verb|value| - (string) символ, обозначающий пустое слово. Значение по умолчанию: \verb|eps|;

\verb|chance| - (float) вероятность добавления пустого слова к правилу. Значение по умолчанию: \verb|0.35|;

\verb|production|:

\verb|max_symbols| - (int) максимальное количество элементов продукции. Значение по умолчанию: \verb|5|;
\verb|terminals|:

\verb|regex| - (string) регулярное выражение для задания значений терминалов. Значение по умолчанию: \verb|[a-z0-9]|;

\verb|max_length| - (int) максимальная длина терминала. Значение по умолчанию: \verb|4|;

\verb|nonterminals|:

\verb|nonterminal_start| - (string) разделитель для отличия нетерминала от терминала (в начале нетерминала). Значение по умолчанию: \verb|[|;

\verb|nonterminal_end| - (string) разделитель для отличия нетерминала от терминала (в конце нетерминала). Значение по умолчанию: \verb|]|;

\verb|regex| - (string) регулярное выражение для задания значений  нетерминалов. Значение по умолчанию: \verb|[A-Z]|;

\verb|max_length| - (int) максимальная длина нетерминала. Значение по умолчанию: \verb|4|;

\section{Примеры работы}

\subsection{Стандартное использование}
Запустим генератор, используя конфигурационный файл по-умолчанию:

\begin{lstlisting}
$ python generator.py < configs/default.yaml
[H] -> [P]6g[H] | [J][O][H];
[O] -> [O]k4[H][H] | [S]n;
[P] -> [S][S]5[J] | [J] | epsilon;
[S] -> [S] | [P];
[J] -> [O][O][S][J];

is generating: False
all reachable: True
\end{lstlisting}

Попробуем изменить параметр \verb|production_separator| в конфигурационном файле на \verb|;| и включить проверку на LL(1):

\begin{lstlisting}
$ python generator.py --check-ll1 < configs/default.yaml
Traceback (most recent call last):
  File ...
  ...
validator.NotLL1Grammar: For LL1 Grammar rule_separator
should be different from production_separator
\end{lstlisting}

\subsection{Продвинутое использование}

Объявим класс генератора \verb|Generator| и попробуем сгенерировать грамматику. \verb|Generator| инициализируется конфигом, который можно считать, используя библиотеку \verb|pyyaml|.

\begin{lstlisting}
from generator import Generator
from cfg import CFG

gen = Generator(config)
cfg = gen.generate_grammar()
\end{lstlisting}

\verb|cfg| - экземпляр класса \verb|CFG|. Этот класс можно расширить, добавив методы для анализа и обработки КС-грамматик. Работа с классом \verb|CFG| абстрагирована от пользовательского синтаксиса, при этом \verb|print(cfg)| выведет грамматику в пользовательском синтаксисе.

\end{document}